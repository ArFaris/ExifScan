\documentclass[a4paper,12pt]{article}
\usepackage[T2A]{fontenc}
\usepackage[utf8]{inputenc}
\usepackage[russian]{babel}
\usepackage{geometry}
\usepackage{hyperref}
\usepackage{graphicx}
\usepackage{float}
\usepackage{caption}
\usepackage{tabularx}
\usepackage{enumitem}
\usepackage{booktabs}

\geometry{
    a4paper,
    left=25mm,
    right=15mm,
    top=20mm,
    bottom=20mm
}

\begin{document}

\begin{titlepage}
    \centering
    \vspace*{2cm}
    
    {\Huge\textbf{EXIFSCAN}}\\[0.5cm]
    {\Large Анализатор метаданных фотографий}\\[1cm]
    
    \vspace{1cm}
    {\Huge\textbf{Документация и руководство пользователя}}\\[0.5cm]
    
    \vspace{1.5cm}
    {\large Выполнила: Фарисеева Арина Дмитриевна}\\
    {\large Группа: ИКБО-31-24}\\
    {\large Дата: \today}\\
    
    \vfill
\end{titlepage}

\tableofcontents
\newpage

\section{Введение}
ExifScan — веб-приложение для анализа метаданных цифровых фотографий в формате EXIF (Exchangeable Image File Format). Приложение позволяет извлекать информацию о параметрах съемки, оборудовании и геолокации из цифровых изображений.

\section{Установка и запуск}

\subsection{Требования}
\begin{itemize}
    \item Браузер: Chrome 90+, Firefox 88+, Edge 90+ (с поддержкой File API)
    \item Node.js версии 16+ (для тестирования и разработки)
    \item Git для клонирования репозитория
\end{itemize}

\subsection{Шаги установки}
\begin{enumerate}
    \item Клонирование репозитория:
    
    \texttt{git clone https://github.com/ArFaris/ExifScan.git}
    
    \texttt{cd ExifScan}

    \item Установка зависимостей:
    
    \texttt{npm install}

    \item Запуск тестов:
    
    \texttt{npm test}

    \item Запуск локального сервера:
    
    \texttt{python -m http.server 8000}
    
    или
    
    \texttt{npx serve .}
\end{enumerate}

\section{UML-диаграммы}

На рисунке \ref{fig:class-diagram} представлена диаграмма классов приложения ExifScan, показывающая взаимосвязи между основными компонентами системы.

\begin{figure}[H]
    \centering
    \includegraphics[width=0.8\textwidth]{../assets/img/doc/image.png}
    \caption{Диаграмма классов приложения ExifScan}
    \label{fig:class-diagram}
\end{figure}

На рисунке \ref{fig:use-case-diagram} показана диаграмма вариантов использования, иллюстрирующая взаимодействие пользователя с системой.

\begin{figure}[H]
    \centering
    \includegraphics[width=0.8\textwidth]{../assets/img/doc/image-1.png}
    \caption{Диаграмма вариантов использования ExifScan}
    \label{fig:use-case-diagram}
\end{figure}

\section{Руководство пользователя}

\subsection{Интерфейс приложения}
\begin{enumerate}
    \item Зона загрузки файлов с поддержкой Drag\&Drop
    \item Кнопка выбора файла через диалоговое окно
    \item Кнопка анализа метаданных
    \item Область отображения результатов, сгруппированных по категориям
\end{enumerate}

\subsection{Шаги работы}
\begin{enumerate}
    \item Загрузите JPEG или PNG файл (до 10 МБ)
    \item Нажмите кнопку "Получить метаданные"
    \item Просмотрите результаты, сгруппированные по категориям:
    \begin{itemize}
        \item Информация о камере
        \item Настройки съемки
        \item Временные метки
        \item GPS координаты (если доступны)
        \item Техническая информация о файле
    \end{itemize}
\end{enumerate}

\section{Документация frontend-компонентов}

\subsection{ExifParser}
Основной класс для извлечения EXIF метаданных. Реализует паттерн "Модель" в архитектуре MVC приложения ExifScan.

\texttt{const parser = new ExifParser(file);}

\texttt{const metadata = await parser.parse();}

\begin{verbatim}
// Структура возвращаемых данных
{
    camera: { ... },      // Камера и объектив
    settings: { ... },    // Настройки съемки  
    time: { ... },        // Даты и время
    gps: { ... },         // Координаты
    fileInfo: { ... }     // Технические данные файла
}
\end{verbatim}

\subsection{UploadManager}
Управление загрузкой файлов, реализует валидацию и обработку событий Drag\&Drop.

\begin{verbatim}
class UploadManager {
    validateFile(file) {
        // Проверка типа и размера файла
        // Поддерживаются JPEG/PNG до 10 МБ
    }
    
    handleDrop(e) {
        // Обработка события перетаскивания файла
    }

    async analyzeFile() {
        // Координация процесса анализа метаданных
    }
}
\end{verbatim}

\section{Структура проекта}

В таблице \ref{tab:project-structure} представлена структура проекта ExifScan.

\begin{table}[H]
\centering
\caption{Структура проекта ExifScan}
\label{tab:project-structure}
\begin{tabular}{|l|p{0.7\textwidth}|}
\hline
\textbf{Файл/Папка} & \textbf{Назначение} \\ 
\hline
index.html & Главная страница приложения \\ 
\hline
src/app.js & Основной скрипт приложения \\ 
\hline
src/managers/ & Менеджеры приложения (Upload, UI, Display, Page) \\ 
\hline
src/parsers/ExifParser.js & Парсер EXIF метаданных \\ 
\hline
assets/ & Статические ресурсы (иконки, изображения) \\ 
\hline
README.md & Основная документация проекта \\ 
\hline
package.json & Конфигурация npm и зависимости \\ 
\hline
jest.config.js & Конфигурация тестового фреймворка Jest \\ 
\hline
\end{tabular}
\end{table}

\section{Тестирование}

\subsection{Тестовая конфигурация}
\begin{verbatim}
// jest.config.js
export default {
    testEnvironment: 'jsdom',
    transform: {} 
};
\end{verbatim}

\subsection{Запуск тестов}
\texttt{npm test}

\section{Библиография}

\begin{thebibliography}{9}

\bibitem{exif-standard} 
Japan Electronics and Information Technology Industries Association (JEITA). 
\emph{Exchangeable image file format for digital still cameras: Exif Version 2.32}. 
Standard CP-3451C, 2019.

\bibitem{exifreader-library}
Wallander, M.
\emph{ExifReader: A JavaScript library for reading EXIF metadata from image files}. 
GitHub repository, 2023. 
Доступно: \url{https://github.com/mattiasw/ExifReader}

\bibitem{javascript-guide}
Flanagan, D.
\emph{JavaScript: The Definitive Guide}. 
7-е изд., O'Reilly Media, 2020.

\bibitem{web-file-api}
World Wide Web Consortium (W3C).
\emph{File API Specification}. 
W3C Working Draft, 2021. 
Доступно: \url{https://www.w3.org/TR/FileAPI/}

\bibitem{jest-docs}
Meta Open Source.
\emph{Jest: Delightful JavaScript Testing}. 
Документация, 2023. 
Доступно: \url{https://jestjs.io/}

\bibitem{uml-specification}
Object Management Group (OMG).
\emph{Unified Modeling Language (OMG UML) Version 2.5.1}. 
OMG Document Number: formal/2017-12-05, 2017.

\bibitem{latex-companion}
Mittelbach, F., Goossens, M.
\emph{The LaTeX Companion}. 
2-е изд., Addison-Wesley, 2004.

\bibitem{mvc-pattern}
Gamma, E., Helm, R., Johnson, R., Vlissides, J.
\emph{Design Patterns: Elements of Reusable Object-Oriented Software}. 
Addison-Wesley, 1994.

\end{thebibliography}

\section{Заключение}

ExifScan предоставляет удобный интерфейс для анализа EXIF метаданных фотографий. Приложение обладает следующими преимуществами:

\begin{itemize}
    \item \textbf{Простота использования}: Интуитивно понятный интерфейс с поддержкой Drag\&Drop
    \item \textbf{Кросс-платформенность}: Работает в любом современном браузере
    \item \textbf{Модульность архитектуры}: Четкое разделение ответственности между компонентами
    \item \textbf{Полная документация}: Включая техническую документацию и руководство пользователя
    \item \textbf{Открытый исходный код}: Проект доступен на GitHub для изучения и улучшения
\end{itemize}

Приложение успешно решает задачу извлечения и анализа EXIF метаданных, предоставляя пользователям ценную информацию о параметрах съемки и технических характеристиках цифровых фотографий.

\end{document}